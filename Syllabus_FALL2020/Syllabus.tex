% Don't touch this %%%%%%%%%%%%%%%%%%%%%%%%%%%%%%%%%%%%%%%%%%%
\documentclass[12pt]{article}
\usepackage{fullpage}
\usepackage[left=1in,top=1in,right=1in,bottom=1in,headheight=3ex,headsep=3ex]{geometry}
\usepackage{graphicx}
\usepackage{float}
\usepackage{array}


\newcommand{\blankline}{\quad\pagebreak[2]}
%%%%%%%%%%%%%%%%%%%%%%%%%%%%%%%%%%%%%%%%%%%%%%%%%%%%%%%%%%%%%%

% Modify Course title, instructor name, semester here %%%%%%%%

\title{PHYaa 250: Mechanics, Fluids, Waves and Light}
\author{Fall 202}
\date{}

%%%%%%%%%%%%%%%%%%%%%%%%%%%%%%%%%%%%%%%%%%%%%%%%%%%%%%%%%%%%%%

% Don't touch this %%%%%%%%%%%%%%%%%%%%%%%%%%%%%%%%%%%%%%%%%%%
\usepackage[sc]{mathpazo}
%\linespread{1.05} % Palatino needs more leading (space between lines)
\usepackage[T1]{fontenc}
\usepackage[mmddyyyy]{datetime}% http://ctan.org/pkg/datetime
\usepackage{advdate}% http://ctan.org/pkg/advdate
\newdateformat{syldate}{\twodigit{\THEMONTH}/\twodigit{\THEDAY}}
\newsavebox{\MONDAY}\savebox{\MONDAY}{Mon}% Mon
\newcommand{\week}[1]{%
%  \cleardate{mydate}% Clear date
% \newdate{mydate}{\the\day}{\the\month}{\the\year}% Store date
  \paragraph*{\kern-2ex\quad #1, \syldate{\today} - \AdvanceDate[4]\syldate{\today}:}% Set heading  \quad #1
%  \setbox1=\hbox{\shortdayofweekname{\getdateday{mydate}}{\getdatemonth{mydate}}{\getdateyear{mydate}}}%
  \ifdim\wd1=\wd\MONDAY
    \AdvanceDate[7]
  \else
    \AdvanceDate[7]
  \fi%
}
%\usepackage{setspace}
\usepackage{multicol}
%\usepackage{indentfirst}
\usepackage{fancyhdr,lastpage}
\usepackage{url}
\pagestyle{fancy}
\usepackage{hyperref}
\usepackage{lastpage}
\usepackage{amsmath}
\usepackage{layout}

\lhead{}
\chead{}
%%%%%%%%%%%%%%%%%%%%%%%%%%%%%%%%%%%%%%%%%%%%%%%%%%%%%%%%%%%%%%

% Modify header here %%%%%%%%%%%%%%%%%%%%%%%%%%%%%%%%%%%%%%%%%
\rhead{\footnotesize Text in header}

%%%%%%%%%%%%%%%%%%%%%%%%%%%%%%%%%%%%%%%%%%%%%%%%%%%%%%%%%%%%%%
% Don't touch this %%%%%%%%%%%%%%%%%%%%%%%%%%%%%%%%%%%%%%%%%%%
\lfoot{}
\cfoot{\small \thepage/\pageref*{LastPage}}
\rfoot{}

\usepackage{array, xcolor}
\usepackage{color,hyperref}
\definecolor{clemsonorange}{HTML}{EA6A20}
\hypersetup{colorlinks,breaklinks,linkcolor=clemsonorange,urlcolor=clemsonorange,anchorcolor=clemsonorange,citecolor=black}

\begin{document}

\maketitle

%\blankline

%\begin{tabular*}{.93\textwidth}{@{\extracolsep{\fill}}lr}

%%%%%%%%%%%%%%%%%%%%%%%%%%%%%%%%%%%%%%%%%%%%%%%%%%%%%%%%%%%%%%

% Modify information %%%%%%%%%%%%%%%%%%%%%%%%%%%%%%%%%%%%%%%%%
%E-mail: \texttt{anabela.turlione@digipen.edu}  \\

 %Office Hours: M 10-11:45am  &  Class Hours: T/Th 3-4:15pm \\

 %Office: ... & Class Room: ... \\
%% & \\
%Lab Room: ... & Lab Hours: W 3-5pm \\
%&  \\
%\hline
%\end{tabular*}

%\begin{figure*}
%\includegraphics[width=1.3\textwidth,angle=90]{Concept_map_315.pdf}
%\end{figure*}


\hrule



% First Section %%%%%%%%%%%%%%%%%%%%%%%%%%%%%%%%%%%%%%%%%%%%
\section*{General Information }

Class Schedule: Tuesday/Thursday 15:30 – 17:25 \\ 
\\
Class room: Ada Byron\\
\\
Professor: Anabela Turlione\\
\\
Contact: anabela.turlione@digipen.edu - int:1029\\
\\
Class web page: PHY250 at distance.digipen.edu\\
\\
Office hours: by appointment\\

\section*{Prerequisites }
 PHY200 Motion dynamics and MAT200 Calculus and Analytical Geometry II.\\
\\
It will be assumed that the student has knowledge in Kinematics, Newtonian Dynamics, work and the law of conservation of 
energy.	
It will also be assumed that the student has some basic knowledge in calculus of derivatives, 
basic trigonometrical curves and operations, common geometrical relations, integral calculus and power series. 
The students should revisit the above topics before entering in deeper physics.

%\bigskip

%\noindent New paragraph. Bla bla bla ...

% Second Section %%%%%%%%%%%%%%%%%%%%%%%%%%%%%%%%%%%%%%%%%%%

\section*{Description}

This calculus-based course provides a fundamental understanding of 
clasical mechanics, fluid dynamics, oscillations, waves and optics. 
Attention will be paid to numerical applications that are relevant to simulations.  


%\begin{itemize}
%\item Course notes available on Moodle. Books. Tech. Bla bla bla ...
%\end{itemize}

% Third Section %%%%%%%%%%%%%%%%%%%%%%%%%%%%%%%%%%%%%%%%%%%

\section*{Course Objectives and Learning Outcomes }
Upon a successfully completion of this course the students will gain a fundamental understanding of :

\begin{enumerate}
\item System of particles, particle collision. Rigid bodies. Rotational Kinetics and dynamics.
\item  Main characteristic of fluids which will allow them to solve buoyancy problems and describe laminar flow with and without viscosity.
\item  The causes of the oscillations and its mathematical description. The students will be able to find the frequency of simple and physical pendula 
as well as of an oscillating elastic body.
\item  Creation and propagation of mechanical waves and its description through the wave equation.
\item  Interferences between traveling waves, and their interferences templates. 
\item  The phenomena of standing waves and resonance. The students will learn to find the nodes and the harmonic frequencies of resonance. 
\item The sound and the common effects of its propagation, Doppler effect, beats, etc.
\item  Ray model of light, the foundations of ray tracers and the dual nature of the light.

\end{enumerate}

% Fourth Section %%%%%%%%%%%%%%%%%%%%%%%%%%%%%%%%%%%%%%%%%%%

\section*{Textbooks}

\begin{itemize}
\item 
Physics for Scientists and Engineers, Prentice Hall 4th Edition, by Giancoli 
		Chapters:  13-16, 32-34, 17-20  
		ISBN: 0 – 13 – 149508 – 9 
\item Physics for Scientists and Engineers: Foundations and Connections with Modern Physics,
		by Deborah M. Katz, Cengage Learning 2017.  
		Chapters:  9, 15-22, 35-38. 
		ISBN: 978-1-305-85698-1
\end{itemize}

\section*{Optional Textbooks}

University Physics, Addison-Wesley 9th Edition, by Young/Freedman
			ISBN: 0 – 201 – 57157 – 9 

\section*{Relation with other subjects}

PHY250 closes PHY300. 
% Fifth Section %%%%%%%%%%%%%%%%%%%%%%%%%%%%%%%%%%%%%%%%%%%

\section*{Grading Policy}

The breakdown of the weighting of the Total Score will be as follows:

\begin{enumerate}
\item  	Homeworks (30 $\%$)
\item   Mid term (30$\%$)
\item 	Final exam (40$\%$)

\end{enumerate}

The minimum grade to pass the subject is 60 \% 


\section*{Mechanisms and Procedures}

\begin{enumerate}
\item Before the class: To optimize the learning experience and the efficient use of our time, reading the relevant book sections before the material is taught
 is recommended. 
\item Attendance:  attendance in class is mandatory. If you are absent for 2 weeks or more you are considered to have withdrawn from the course. If you decide to
 drop the course it is your responsibility to follow the correct procedures.
\item No food is allowed in class. I strongly recommend to take notes during the class.
\item Working problems is essential in mastering the material. There will be approximately an assignment every two weeks.  At least, one of the assigments will be a 
programming assignments. Programming assignments must be submitted to Moodle before the deadline.
\item Late Policy: Late homework will not be accepted.
\item  Please fell free to send me an email whenever you need help.
\item A calculator is required for tests and homework. You can use any programming calculator without an Internet connection during any test or exam.
\item All exams are closed book. As this course is about understanding and not memorization, one sheet of notes is permitted for an exam with the formulas you 
consider. This sheet of notes must be handwritten by you, and no larger than a ‘normal’ (DIN4) piece of paper. Front and back of the page may be used.
\item Missing a test without a timely, valid excuse will result in a 0 score for the test. There are NO make up exams unless you have a compelling and well 
documented reason for missing a test. Notice that make ups are only considered under relevant medical, familiar or administrative situations that cannot be 
postponed.
\end{enumerate}

\section*{Rubrics and Assessment}

To get full credit you need to show all the important steps of your work which are:

\begin{enumerate}
\item Do a drawing of the schedule/body diagram in each problem.
\item Indicate the law/theory applied (or your reasoning).
\item Develop your calculus. The level of detail required is what your colleagues would need to see in order to understand your solution, without having to 
work it out for themselves.
\item Give the solution, specifying the unit of each magnitude and the direction of them (it's a vector). A penalty of 10% will be apply if the units are missing. 
\end{enumerate}

\begin{itemize}
\item Before submitting an assignment your grade is a 0, not a 100. This means that you obtain points for doing things right, and I do not subtract points from 
that non existing 100. 
\item Partial credit will be given only if your work is clearly presented and mostly correct.
\item If the process that you follow is correct but you arrive to a conclusion that is totally inconsistent with the theory learned in class, you will get a zero
 in that exercises with the note “misconception” attached. 
\item If an error is accumulative along an exercise, that will not penalize the rest of the exercise unless this means inconsistency with the theory learned in 
class. 
\item Any material covered in the course is valid for testing, including concepts covered in lecture, homework, or other communications and/or assigned work 
(as reading the textbook).
\item No messy tests/homework will be graded.
\end{itemize}

\section*{Relevance/Statement}

It is important to keep up with the material, to study regularly at home (at least 2 hours for every hour in class) and to do as many problems as you can 
(don’t limit yourself to the assigned or recommended problems, or merely the problems that are due).  

	You are welcome to work with other students, so long as the aim is furthering your understanding of the concepts and problem solving techniques. I am 
	happy to help work through problems, either in office hours or in class. Just remember, doing a problem yourself is very different from watching another 
	person do so. If you work together on problem sets, be sure to provide your own solution to every problem proving that you understand your writing. 
	In addition, some exam questions may be a resemblance to homework questions, so you’re encouraged to fully understand what you turn in. 
	Again, reading the relevant book sections before the material is taught is highly recommended. 


\subsection*{Last Day to Withdraw:    1st November}

\vspace{5 mm}

\section*{Academic Integrity Policy}

Academic dishonesty in any form will not be tolerated in this course. Cheating, copying from any sources (including current or past students work, online sources 
or books), plagiarizing, or any other form of academic dishonesty (including doing someone else’s individual assignments) will result in, at the extreme minimum,
 a zero on the assignment in question, and could result in a failing grade in the course or even expulsion from DigiPen. Assisting others in cheating is prohibited 
 and will be equally punished.

\section*{Disability Support Services}

If students have disabilities and will need formal accommodations in order to fully participate or effectively demonstrate learning in this class, they should 
contact the Administration Office at 946365163. The Administration Office welcomes the opportunity to meet with students to discuss how the accommodations will
 be implemented. Also, if you may need assistance in the event of an evacuation, please let the instructor know. 

%\subsection*{Class Structure}

%Bla bla bla ...

%\bigskip

%Bla bla bla ...

%\subsection*{Assessments}

%...

%\subsubsection*{Lecture}
%Bla bla ...

%\subsubsection*{Lab}
%...

%\subsubsection*{Final Exam and Class Project}

%Bla bla \textbf{Bla bla}.

%\subsection*{Grading Policy}
%The typical NC State grading scale will be used. I reserve the right to curve the scale dependent on overall class scores at the end of the semester. Any curve will only ever make it easier to obtain a certain letter grade. The grade will count the assessments using the following proportions:
%\begin{itemize}
%	\item \underline{\textbf{30\%}} of your grade will be determined by 2 in class midterm exams (15\% each).
%	\item \underline{\textbf{5\%}} of your grade will be determined by ...
%	\item \underline{\textbf{5\%}} ...
%    \item \underline{\textbf{10\%}}  ...
%	\item \underline{\textbf{15\%}} ...
%	\item \underline{\textbf{15\%}} ...
%\end{itemize}

% Add a figure %%%%%%%%%%%%%%%%%%%%%%%%%%%%%%%%%%%%%%%%%%%

%\begin{figure*}
%\includegraphics[width=1.3\textwidth,angle=90]{Concept_map_315.pdf}
%\end{figure*}

% Fifth Section %%%%%%%%%%%%%%%%%%%%%%%%%%%%%%%%%%%%%%%%%%%

%\newpage



% Course Schedule %%%%%%%%%%%%%%%%%%%%%%%%%%%%%%%%%%%%%%%%%%%

\section*{Outline and Tentative Dates}


\raggedright
\begin{center}

 \begin{tabular}{|c |l| l| l|} 


 \hline
Timeline & Topic & HW &\multicolumn{1}{p{3cm}|}{ Approximate book Section} \\ [0.5ex] 
 \hline\hline
 Week 1 &\multicolumn{1}{p{10cm}|}{ Conservation of Linear Momentum. Impulse and Collisions} &  &Ch 9  \\ 
 \hline
  Week 2 & \multicolumn{1}{p{10cm}|}{Center of Mass (CM). Translation of CM}   &HW 1 &Ch 9  \\
 \hline
 Week 3 & Rotational Kinematics. Torque and Rotational Dynamics & & Ch 10-11 \\
 \hline
 Week 4 & \multicolumn{1}{p{10cm}|}{Rigid Bodies and Inertia}  & HW 2 & Ch 11 \\
 \hline

 Week 5 & \multicolumn{1}{p{10cm}|}{ Rolling. Angular Momentum  Conservation of Angular Momentum } & & Ch 11  \\
\hline

 Week 6 & \multicolumn{1}{p{10cm}|}{  Static Equilibrium, Elasticity } &  HW 3& Ch 12  \\
\hline

 Week 7 &\multicolumn{1}{p{10cm}|}{ 
Fluids mechanics: Density, Pressure and Pascal Principle,
Buoyancy and Archimedes Principle} 
 &  & Ch 13 \\ 

 \hline
 Week 8 &\multicolumn{1}{p{10cm}|}{ Flow, Equation of  Continuity, Bernoulli's equation
Surface Tension, Viscosity, Poiseuille's equation, Drag forces \textbf{MIDTERM} }&  HW 4 &Ch 13  \\ 
 \hline
 Week 9 & \multicolumn{1}{p{10cm}|}{ Fluids
Oscillations: SHM (Math equation, change of phase)}
 & & Ch 14 \\ 
 \hline
 Week 10 &\multicolumn{1}{p{10cm}|}{ Energy of SHM, Potential Diagram. Simple Pendulum.  Physical Pendulum.
Oscillations in 2D, DHM, Resonance} & HW 5 &Ch 14  \\ 
 \hline
 Week 11 &\multicolumn{1}{p{10cm}|}{ Waves: Wave Motion,  Mathematical Representation 
Traveling Waves, Longitudinal and Transversal Waves}&  & Ch 15 \\ 
 \hline
 Week 12 &\multicolumn{1}{p{10cm}|}{ Energy on a Wave. Principle of Superposition.
Interference, Standing Waves}  & HW 6 & Ch 15 \\ 
 \hline
 Week 13 & \multicolumn{1}{p{10cm}|}{Sources of sound: Resonance. 
Sound I: Intensity (dB), Sound Level, Ear Response. Beats. Sound II: Doppler Effect, Shocks Waves.} &  & Ch 16 \\ 
 \hline
 Week 14 &\multicolumn{1}{p{10cm}|}{
Light Reflection: Plane Mirror, Spherical Mirror. Light Refraction: Snell's Laws, spherical refraction.
Thin lenses, ray tracing. Difraction. } &  HW 7  & Ch 16, Ch 32 \\ 
 \hline

 Week 15 & \textbf{FINAL EXAM}&  &  \\ 
 \hline
\end{tabular}

\end{center}


[This entire syllabus, particularly the time line, may be adjusted or changed at any time by the instructor.]

\end{document}


