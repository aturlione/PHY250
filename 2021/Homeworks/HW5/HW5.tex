% Don't touch this %%%%%%%%%%%%%%%%%%%%%%%%%%%%%%%%%%%%%%%%%%%
\documentclass[12pt]{article}
\usepackage{fullpage}
\usepackage[left=1in,top=1in,right=1in,bottom=1in,headheight=3ex,headsep=3ex]{geometry}
\usepackage{graphicx}
\usepackage{float}
\usepackage{array}


\newcommand{\blankline}{\quad\pagebreak[2]}
%%%%%%%%%%%%%%%%%%%%%%%%%%%%%%%%%%%%%%%%%%%%%%%%%%%%%%%%%%%%%%

% Modify Course title, instructor name, semester here %%%%%%%%

\title{Homework 5: 2D waves}
\author{PHY250 - Fall 2021}
\date{}

%%%%%%%%%%%%%%%%%%%%%%%%%%%%%%%%%%%%%%%%%%%%%%%%%%%%%%%%%%%%%%

% Don't touch this %%%%%%%%%%%%%%%%%%%%%%%%%%%%%%%%%%%%%%%%%%%
\usepackage[sc]{mathpazo}
%\linespread{1.05} % Palatino needs more leading (space between lines)
\usepackage[T1]{fontenc}
\usepackage[mmddyyyy]{datetime}% http://ctan.org/pkg/datetime
\usepackage{advdate}% http://ctan.org/pkg/advdate

\usepackage{setspace}

\newcommand{\HRule}{\rule{\linewidth}{0.5mm}}
\newdateformat{syldate}{\twodigit{\THEMONTH}/\twodigit{\THEDAY}}
\newsavebox{\MONDAY}\savebox{\MONDAY}{Mon}% Mon
\newcommand{\week}[1]{%
%  \cleardate{mydate}% Clear date
% \newdate{mydate}{\the\day}{\the\month}{\the\year}% Store date
  \paragraph*{\kern-2ex\quad #1, \syldate{\today} - \AdvanceDate[4]\syldate{\today}:}% Set heading  \quad #1
%  \setbox1=\hbox{\shortdayofweekname{\getdateday{mydate}}{\getdatemonth{mydate}}{\getdateyear{mydate}}}%
  \ifdim\wd1=\wd\MONDAY
    \AdvanceDate[7]
  \else
    \AdvanceDate[7]
  \fi%
}
%\usepackage{setspace}
\usepackage{multicol}
%\usepackage{indentfirst}
\usepackage{fancyhdr,lastpage}
\usepackage{url}
\pagestyle{fancy}
\usepackage{hyperref}
\usepackage{lastpage}
\usepackage{amsmath}
\usepackage{layout}

\lhead{}
\chead{}
%%%%%%%%%%%%%%%%%%%%%%%%%%%%%%%%%%%%%%%%%%%%%%%%%%%%%%%%%%%%%%

% Modify header here %%%%%%%%%%%%%%%%%%%%%%%%%%%%%%%%%%%%%%%%%
%\rhead{\footnotesize Text in header}

%%%%%%%%%%%%%%%%%%%%%%%%%%%%%%%%%%%%%%%%%%%%%%%%%%%%%%%%%%%%%%
% Don't touch this %%%%%%%%%%%%%%%%%%%%%%%%%%%%%%%%%%%%%%%%%%%
\lfoot{}
\cfoot{\small \thepage/\pageref*{LastPage}}
\rfoot{}

\usepackage{array, xcolor}
\usepackage{color,hyperref}
\definecolor{clemsonorange}{HTML}{EA6A20}
\hypersetup{colorlinks,breaklinks,linkcolor=clemsonorange,urlcolor=clemsonorange,anchorcolor=clemsonorange,citecolor=black}



\begin{document}


\maketitle

\textcolor{red}{Deadline: 11/29/2021}

\vspace{5mm}

\begin{spacing}{0.3}
    \noindent
    \HRule\\
    \HRule
\end{spacing}
\vspace{5mm}


% First Section %%%%%%%%%%%%%%%%%%%%%%%%%%%%%%%%%%%%%%%%%%%%


\newcounter{example}
\setcounter{example}{1}

\section*{Exercise \theexample: Traveling waves }
Simulate in Octave a traveling wave in 2D  with an angular frequency $\omega=1$ and amplitude $A=1$ considering the following cases:

\begin{enumerate}
  \item $\vec{k}=(2,0)$.
  \item $\vec{k}=2\cdot\hat{e}_r$, where $\hat{e}_r$ is the radial direction in circular coordinates.
  \item $\vec{k}=(2,2)$.
\end{enumerate}

% First Section %%%%%%%%%%%%%%%%%%%%%%%%%%%%%%%%%%%%%%%%%%%%



\stepcounter{example}

\section*{Exercise \theexample: Standing waves }
Simulate in Octave standing waves in a plaque of dimensions 1x1, generated by traveling waves with angular frequency $\omega=1$ and amplitude $A=1$. Plot at least three different harmonics.

% First Section %%%%%%%%%%%%%%%%%%%%%%%%%%%%%%%%%%%%%%%%%%%%



\stepcounter{example}

\section*{Exercise \theexample: Interference }
Simulate in Octave the interference pattern produced by 2 sources in water separated by a distance $d=10$ and an angular frequency $\omega=1$.
\begin{enumerate}
  \item choose one of the three cases of the exercise 2 to simulate the sources. Which one is the right choice? why?
  \item What happens when you increase the distance $d$?
  \item What happens when you modify the frequency of one of the sources?
\end{enumerate}

\end{document}


