% Don't touch this %%%%%%%%%%%%%%%%%%%%%%%%%%%%%%%%%%%%%%%%%%%
\documentclass[12pt]{article}
\usepackage{fullpage}
\usepackage[left=1in,top=1in,right=1in,bottom=1in,headheight=3ex,headsep=3ex]{geometry}
\usepackage{graphicx}
\usepackage{float}
\usepackage{array}


\newcommand{\blankline}{\quad\pagebreak[2]}
%%%%%%%%%%%%%%%%%%%%%%%%%%%%%%%%%%%%%%%%%%%%%%%%%%%%%%%%%%%%%%

% Modify Course title, instructor name, semester here %%%%%%%%

\title{PHY 250: FINAL PROJECT}
\author{Fall 2020}
\date{}

%%%%%%%%%%%%%%%%%%%%%%%%%%%%%%%%%%%%%%%%%%%%%%%%%%%%%%%%%%%%%%

% Don't touch this %%%%%%%%%%%%%%%%%%%%%%%%%%%%%%%%%%%%%%%%%%%
\usepackage[sc]{mathpazo}
%\linespread{1.05} % Palatino needs more leading (space between lines)
\usepackage[T1]{fontenc}
\usepackage[mmddyyyy]{datetime}% http://ctan.org/pkg/datetime
\usepackage{advdate}% http://ctan.org/pkg/advdate
\newdateformat{syldate}{\twodigit{\THEMONTH}/\twodigit{\THEDAY}}
\newsavebox{\MONDAY}\savebox{\MONDAY}{Mon}% Mon
\newcommand{\week}[1]{%
%  \cleardate{mydate}% Clear date
% \newdate{mydate}{\the\day}{\the\month}{\the\year}% Store date
  \paragraph*{\kern-2ex\quad #1, \syldate{\today} - \AdvanceDate[4]\syldate{\today}:}% Set heading  \quad #1
%  \setbox1=\hbox{\shortdayofweekname{\getdateday{mydate}}{\getdatemonth{mydate}}{\getdateyear{mydate}}}%
  \ifdim\wd1=\wd\MONDAY
    \AdvanceDate[7]
  \else
    \AdvanceDate[7]
  \fi%
}
%\usepackage{setspace}
\usepackage{multicol}
%\usepackage{indentfirst}
\usepackage{fancyhdr,lastpage}
\usepackage{url}
\pagestyle{fancy}
\usepackage{hyperref}
\usepackage{lastpage}
\usepackage{amsmath}
\usepackage{layout}

\lhead{}
\chead{}
%%%%%%%%%%%%%%%%%%%%%%%%%%%%%%%%%%%%%%%%%%%%%%%%%%%%%%%%%%%%%%

% Modify header here %%%%%%%%%%%%%%%%%%%%%%%%%%%%%%%%%%%%%%%%%
%\rhead{\footnotesize Text in header}

%%%%%%%%%%%%%%%%%%%%%%%%%%%%%%%%%%%%%%%%%%%%%%%%%%%%%%%%%%%%%%
% Don't touch this %%%%%%%%%%%%%%%%%%%%%%%%%%%%%%%%%%%%%%%%%%%
\lfoot{}
\cfoot{\small \thepage/\pageref*{LastPage}}
\rfoot{}

\usepackage{array, xcolor}
\usepackage{color,hyperref}
\definecolor{clemsonorange}{HTML}{EA6A20}
\hypersetup{colorlinks,breaklinks,linkcolor=clemsonorange,urlcolor=clemsonorange,anchorcolor=clemsonorange,citecolor=black}

\begin{document}

\maketitle

%\blankline

%\begin{tabular*}{.93\textwidth}{@{\extracolsep{\fill}}lr}

%%%%%%%%%%%%%%%%%%%%%%%%%%%%%%%%%%%%%%%%%%%%%%%%%%%%%%%%%%%%%%

% Modify information %%%%%%%%%%%%%%%%%%%%%%%%%%%%%%%%%%%%%%%%%
%E-mail: \texttt{anabela.turlione@digipen.edu}  \\

 %Office Hours: M 10-11:45am  &  Class Hours: T/Th 3-4:15pm \\

 %Office: ... & Class Room: ... \\
%% & \\
%Lab Room: ... & Lab Hours: W 3-5pm \\
%&  \\
%\hline
%\end{tabular*}

%\begin{figure*}
%\includegraphics[width=1.3\textwidth,angle=90]{Concept_map_315.pdf}
%\end{figure*}

\begin{center}
Death-line: December  18th    
\end{center}
\hrule



% First Section %%%%%%%%%%%%%%%%%%%%%%%%%%%%%%%%%%%%%%%%%%%%



\section*{Submition Instructions}

The project must be done in LaTeX, libreoffice, Word, etc.
\vspace{3mm}
\\
You must submit a zip file to moodle containing:
\vspace{3mm}

\begin{itemize}
\item The Octave codes.
\item The pdf file with the theory development.
\item The animated videos for the collisions.
\item The sound files of exercise 2.
\end{itemize}

\vspace{10mm}

\section*{Exercise A}

Solve a collision between an ellipse of mass $M$ and a particle of mass $m$. 
Consider that the initial inclination of the elipse axis is $0^{\circ}$.


\vspace{10mm}

\begin{enumerate}
\item Find tthe final velocity of the particle $\vec{V}'_1$,
the center of mass of the elipse $\vec{V}'_{CM}$, and the angular velocity $\omega$ in term of 
the initial velocities and the impulse $J$.
\vspace{3mm}
\item Use the deffinition of the restitution coefficient to find the Expresion of $J$ in therm 
of the initial velocities.
\vspace{3mm}
\item Write a code in Octave to calculate the final velocities of a collision for  given initial
conditions.
\vspace{3mm}
\item Plot the solution for different times and build and animation.
\end{enumerate}

\vspace{5mm}

\textbf{Consider the following initial conditions:}

\vspace{3mm}

\begin{itemize}
\item $\vec{v}_1=v_1\hat{\imath}$, $\vec{v}_{CM}=0$, $\omega=0$
\item e=1, e=0
\item  $M=2m$
\end{itemize}

\vspace{5mm}

\textbf{You can structure your code as follows:}


\begin{center}
\begin{tabular}{ l  }
\hline
Geometry\\ 
 \hline
Define Space Mesh\\
 Build the elipse\\
 \hline
 Set the IC for the elipse:\\
  \hline

 Mass\\
 Inertia Moment\\
  CM position\\
 CM velocity$\rightarrow\vec{v}_{CM}=0$\\
 Inclination $\rightarrow\theta_0=0$\\
 \hline

 Set the IC for the particle\\
 \hline

 mass\\
  initial position\\
 initial velocity\\
 \hline

 Set a value for the restitution coefficient\\
  \hline

 START TIME LOOP\\
 \hline

 motion of the particle at constant speed: $\vec{r}_p=\vec{r}_{p0}+\vec{V}_p\Delta t$\\
  linear motion of the elipse: $\vec{r}_{CM}=\vec{r}_{CM}+\vec{V}_{CM}\Delta t$\\
 rotation of the ellipse: $\theta=\theta_0+\omega\Delta t$\\
ellipse translated and rotated\\
 \hline

 DEFINE THE COLLISION CONDITION\\
  \hline

 If the condition is true $\rightarrow$ find contact point\\
  \hline

 Find the normal at the contact point\\
 Find the impulse\\
 Find the final velocities\\
 Find new positions\\
  \hline

\end{tabular}
\end{center}

\newpage
\section*{Exercise B}

\begin{enumerate}
\item Sintetize the sound of a Violin and a Clarinet.

\begin{itemize}
\item  Record the sound with "wavesurfer"\footnote{ WaveSurfer is an open source tool for sound visualization and manipulation.  (\url{https://sourceforge.net/projects/wavesurfer/)
}}  to obtain the spectrum \footnote{Clarinet sound: \url{https://www.youtube.com/watch?v=5Fi97H11KBc}, from 4:39 to 4:40}
\footnote{Violin sound: \url{https://www.youtube.com/watch?v=j0FynYzQvcM}}. 
\vspace{3mm}

\item Extract the intensity corresponding to the first 16 harmonics and make a bars graph.
\vspace{3mm}

\item Find an expresion for the relative amplitude of the  harmonics $A/A_0$ (where $A_0$ is
the amplitude of the first harmonic) in terms of the intensity.

\item Consider that the amplitud of the fundamental frequency is $1$ and obtain the relative amplitudes for
each one of the harmonics.
 
\item Generate a sound wave for those frequencies using Octave.
\vspace{3mm}

\end{itemize}

\item Plot the resultant wave for the violin, the clarinet and the fundamental frequency in 
the same graph.

\vspace{3mm}
\item Compare the spectrum for the violin and the clarinet. which are the main differences?

\vspace{3mm}

\item Compare your sounds with the sources. What are the differences between your signals and thesources?

\vspace{3mm}

\item How could you improve the quality of the digitalized sounds?

\end{enumerate}




%%%%%%%%%%%%%%%%%%%%%%%%%%%%%%%%%%%%%%%%%%%%%%%%%%%%%%%%%%%%%%

\end{document}


%%%%%%%%%%%%%%%%%%%%%%%%%%%%%%%%%%%%%%%%%%%%%%%%%%%%%%%%%%%%%%