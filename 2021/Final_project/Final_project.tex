% Don't touch this %%%%%%%%%%%%%%%%%%%%%%%%%%%%%%%%%%%%%%%%%%%
\documentclass[12pt]{article}
\usepackage{fullpage}
\usepackage[left=1in,top=1in,right=1in,bottom=1in,headheight=3ex,headsep=3ex]{geometry}
\usepackage{graphicx}
\usepackage{float}
\usepackage{array}


\newcommand{\blankline}{\quad\pagebreak[2]}
%%%%%%%%%%%%%%%%%%%%%%%%%%%%%%%%%%%%%%%%%%%%%%%%%%%%%%%%%%%%%%

% Modify Course title, instructor name, semester here %%%%%%%%

\title{Final Project}
\author{PHY250 - Fall 2021}
\date{}

%%%%%%%%%%%%%%%%%%%%%%%%%%%%%%%%%%%%%%%%%%%%%%%%%%%%%%%%%%%%%%

% Don't touch this %%%%%%%%%%%%%%%%%%%%%%%%%%%%%%%%%%%%%%%%%%%
\usepackage[sc]{mathpazo}
%\linespread{1.05} % Palatino needs more leading (space between lines)
\usepackage[T1]{fontenc}
\usepackage[mmddyyyy]{datetime}% http://ctan.org/pkg/datetime
\usepackage{advdate}% http://ctan.org/pkg/advdate

\usepackage{setspace}

\newcommand{\HRule}{\rule{\linewidth}{0.5mm}}
\newdateformat{syldate}{\twodigit{\THEMONTH}/\twodigit{\THEDAY}}
\newsavebox{\MONDAY}\savebox{\MONDAY}{Mon}% Mon
\newcommand{\week}[1]{%
%  \cleardate{mydate}% Clear date
% \newdate{mydate}{\the\day}{\the\month}{\the\year}% Store date
  \paragraph*{\kern-2ex\quad #1, \syldate{\today} - \AdvanceDate[4]\syldate{\today}:}% Set heading  \quad #1
%  \setbox1=\hbox{\shortdayofweekname{\getdateday{mydate}}{\getdatemonth{mydate}}{\getdateyear{mydate}}}%
  \ifdim\wd1=\wd\MONDAY
    \AdvanceDate[7]
  \else
    \AdvanceDate[7]
  \fi%
}
%\usepackage{setspace}
\usepackage{multicol}
%\usepackage{indentfirst}
\usepackage{fancyhdr,lastpage}
\usepackage{url}
\pagestyle{fancy}
\usepackage{hyperref}
\usepackage{lastpage}
\usepackage{amsmath}
\usepackage{layout}

\lhead{}
\chead{}
%%%%%%%%%%%%%%%%%%%%%%%%%%%%%%%%%%%%%%%%%%%%%%%%%%%%%%%%%%%%%%

% Modify header here %%%%%%%%%%%%%%%%%%%%%%%%%%%%%%%%%%%%%%%%%
%\rhead{\footnotesize Text in header}

%%%%%%%%%%%%%%%%%%%%%%%%%%%%%%%%%%%%%%%%%%%%%%%%%%%%%%%%%%%%%%
% Don't touch this %%%%%%%%%%%%%%%%%%%%%%%%%%%%%%%%%%%%%%%%%%%
\lfoot{}
\cfoot{\small \thepage/\pageref*{LastPage}}
\rfoot{}

\usepackage{array, xcolor}
\usepackage{color,hyperref}
\definecolor{clemsonorange}{HTML}{EA6A20}
\hypersetup{colorlinks,breaklinks,linkcolor=clemsonorange,urlcolor=clemsonorange,anchorcolor=clemsonorange,citecolor=black}



\begin{document}


\maketitle

\textcolor{red}{Deadline: 12/17/2021}

\vspace{5mm}

\begin{spacing}{0.3}
    \noindent
    \HRule\\
    \HRule
\end{spacing}
\vspace{5mm}


% First Section %%%%%%%%%%%%%%%%%%%%%%%%%%%%%%%%%%%%%%%%%%%%


\section*{Submission Instructions}

You must submit a zip file to moodle containing:
\vspace{3mm}

\begin{itemize}
\item The Octave codes.
\item The animated video for the fluid.
\item The sound files of exercise 2.
\item Answer the questions using comments in your Octave codes.
\end{itemize}

\vspace{10mm}




\newcounter{example}
\setcounter{example}{1}

\section*{ \theexample- Simulate  flow  passing a sphere}
We are goin to consider  the case of the steady flow of an incompressible, non-viscous, circulation-free liquid passing an sphere of radius $a$ with initial velocity $v_0$.
This is an artificial idealized situation valid for the case of a stretched sheet. Under this considerations, the solution of the Navier-Stokes equations is:\\
\vspace{3mm}
\\
\begin{equation*}
    \vec{v}=\nabla \bigg[-v_0*x\bigg(1+\frac{a^3}{2(x^2+y^2)^{3/2}}\bigg)\bigg]
\end{equation*}
\vspace{6mm}
\\
Re-use the code of Homework 2 to make a simulation for this fluid, considering  $a=0.4$ and $v_0=10$.

\vspace{15mm}
Questions:

\vspace{3mm}

\begin{enumerate}
    \item What does "incompressible, non-viscous, circulation-free liquid" mean? 
    \item What is the difference between this fluid and the one we considered when we solved problems during the course?
    \item Do you think that the volume rate flow is just "$vA$" in this case? Why yes o why not?
    \item How would you improve this model?
\end{enumerate}
% First Section %%%%%%%%%%%%%%%%%%%%%%%%%%%%%%%%%%%%%%%%%%%%

\stepcounter{example}

\section*{ \theexample-  Simulate the sound of different instruments}



    \begin{enumerate}
    \item  Download two different sound sources from  https://theremin.music.uiowa.edu/index.html, convert the files to .wav (you can do it in https://convertio.co/es/).
    \vspace{3mm}
    \item Plot the signal and the spectrum of the sources using Octave.
    \vspace{3mm}
     
    \item Generate a new sound wave considering those frequencies, plot the signal and its spectrum (bars graph) overlapped with the original source plots.
    \vspace{3mm}

    
    \item Compare your sounds with the sources. What are the differences between your signals and the sources?
    
    \vspace{3mm}
    
    \item How could you improve the quality of the digitalized sounds?

    \end{enumerate}
    






\end{document}


