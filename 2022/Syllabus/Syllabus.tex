% Don't touch this %%%%%%%%%%%%%%%%%%%%%%%%%%%%%%%%%%%%%%%%%%%
\documentclass[12pt]{article}
\usepackage{fullpage}
\usepackage[left=1in,top=1in,right=1in,bottom=1in,headheight=3ex,headsep=3ex]{geometry}
\usepackage{graphicx}
\usepackage{float}
\usepackage{array}


\newcommand{\blankline}{\quad\pagebreak[2]}
%%%%%%%%%%%%%%%%%%%%%%%%%%%%%%%%%%%%%%%%%%%%%%%%%%%%%%%%%%%%%%

% Modify Course title, instructor name, semester here %%%%%%%%

\title{PHY 250: Mechanics, Fluids, Waves and Light}
\author{Fall 2022}
\date{}

%%%%%%%%%%%%%%%%%%%%%%%%%%%%%%%%%%%%%%%%%%%%%%%%%%%%%%%%%%%%%%

% Don't touch this %%%%%%%%%%%%%%%%%%%%%%%%%%%%%%%%%%%%%%%%%%%
\usepackage[sc]{mathpazo}
%\linespread{1.05} % Palatino needs more leading (space between lines)
\usepackage[T1]{fontenc}
\usepackage[mmddyyyy]{datetime}% http://ctan.org/pkg/datetime
\usepackage{advdate}% http://ctan.org/pkg/advdate
\newdateformat{syldate}{\twodigit{\THEMONTH}/\twodigit{\THEDAY}}
\newsavebox{\MONDAY}\savebox{\MONDAY}{Mon}% Mon
\newcommand{\week}[1]{%
%  \cleardate{mydate}% Clear date
% \newdate{mydate}{\the\day}{\the\month}{\the\year}% Store date
  \paragraph*{\kern-2ex\quad #1, \syldate{\today} - \AdvanceDate[4]\syldate{\today}:}% Set heading  \quad #1
%  \setbox1=\hbox{\shortdayofweekname{\getdateday{mydate}}{\getdatemonth{mydate}}{\getdateyear{mydate}}}%
  \ifdim\wd1=\wd\MONDAY
    \AdvanceDate[7]
  \else
    \AdvanceDate[7]
  \fi%
}
%\usepackage{setspace}
\usepackage{multicol}
%\usepackage{indentfirst}
\usepackage{fancyhdr,lastpage}
\usepackage{url}
\pagestyle{fancy}
\usepackage{hyperref}
\usepackage{lastpage}
\usepackage{amsmath}
\usepackage{layout}

\lhead{}
\chead{}
%%%%%%%%%%%%%%%%%%%%%%%%%%%%%%%%%%%%%%%%%%%%%%%%%%%%%%%%%%%%%%

% Modify header here %%%%%%%%%%%%%%%%%%%%%%%%%%%%%%%%%%%%%%%%%
\rhead{\footnotesize Text in header}

%%%%%%%%%%%%%%%%%%%%%%%%%%%%%%%%%%%%%%%%%%%%%%%%%%%%%%%%%%%%%%
% Don't touch this %%%%%%%%%%%%%%%%%%%%%%%%%%%%%%%%%%%%%%%%%%%
\lfoot{}
\cfoot{\small \thepage/\pageref*{LastPage}}
\rfoot{}

\usepackage{array, xcolor}
\usepackage{color,hyperref}
\definecolor{clemsonorange}{HTML}{EA6A20}
\hypersetup{colorlinks,breaklinks,linkcolor=clemsonorange,urlcolor=clemsonorange,anchorcolor=clemsonorange,citecolor=black}

\begin{document}

\maketitle

%\blankline

%\begin{tabular*}{.93\textwidth}{@{\extracolsep{\fill}}lr}

%%%%%%%%%%%%%%%%%%%%%%%%%%%%%%%%%%%%%%%%%%%%%%%%%%%%%%%%%%%%%%

% Modify information %%%%%%%%%%%%%%%%%%%%%%%%%%%%%%%%%%%%%%%%%
%E-mail: \texttt{anabela.turlione@digipen.edu}  \\

 %Office Hours: M 10-11:45am  &  Class Hours: T/Th 3-4:15pm \\

 %Office: ... & Class Room: ... \\
%% & \\
%Lab Room: ... & Lab Hours: W 3-5pm \\
%&  \\
%\hline
%\end{tabular*}

%\begin{figure*}
%\includegraphics[width=1.3\textwidth,angle=90]{Concept_map_315.pdf}
%\end{figure*}


\hrule



% First Section %%%%%%%%%%%%%%%%%%%%%%%%%%%%%%%%%%%%%%%%%%%%
\section*{General Information }

Class Schedule:  M 9:30 am - 11:25 pm, W 9:30 am - 11:25 pm\\ 
\\
Class room: 	MARIE S. GERMAIN\\
\\
Professor: Anabela Turlione\\
\\
Contact: anabela.turlione@digipen.edu - int:1029\\
\\
Class web page: PHY250 at distance.digipen.edu\\
\\
Office hours: by appointment\\

\section*{Prerequisites }
 PHY200: Motion dynamics and MAT200 Calculus and Analytical Geometry II.\\
\\
It will be assumed that the student has knowledge in Kinematics, Newtonian Dynamics, work and the law of conservation of 
energy.	
It will also be assumed that the student has some basic knowledge in calculus of derivatives, 
basic trigonometrical curves and operations, common geometrical relations, integral calculus and power series. 
The students should revisit the above topics before entering in deeper physics.

%\bigskip

%\noindent New paragraph. Bla bla bla ...

% Second Section %%%%%%%%%%%%%%%%%%%%%%%%%%%%%%%%%%%%%%%%%%%

\section*{Description}

This  course provides a fundamental understanding of 
clasical mechanics, fluid dynamics, oscillations, waves and optics. 
Attention will be paid to numerical applications that are relevant to simulations.  


%\begin{itemize}
%\item Course notes available on Moodle. Books. Tech. Bla bla bla ...
%\end{itemize}

% Third Section %%%%%%%%%%%%%%%%%%%%%%%%%%%%%%%%%%%%%%%%%%%

\section*{Course Objectives and Learning Outcomes }
Upon a successfully completion of this course the students will gain a fundamental understanding of :

\begin{enumerate}
\item  Main characteristic of fluids which will allow them to solve buoyancy problems and describe laminar flow with and without viscosity.
\item  The causes of the oscillations and its mathematical description. The students will be able to find the frequency of simple and physical pendula 
as well as of an oscillating elastic body.
\item  Creation and propagation of mechanical waves and its description through the wave equation.
\item  Interferences between traveling waves. 
\item  The phenomena of standing waves and resonance. The students will learn to find the nodes and the harmonic frequencies of resonance. 
\item The sound and the common effects of its propagation, Doppler effect, beats, etc.
\item  Ray model of light, the foundations of ray tracers and the dual nature of the light.
\item Understanding the physics of shaders in games. 
\end{enumerate}

% Fourth Section %%%%%%%%%%%%%%%%%%%%%%%%%%%%%%%%%%%%%%%%%%%

\section*{Textbooks}

\begin{itemize}
\item University Physics, 9th Edition, by  Zears $\&$ Zemansky
ISBN: 0 – 201 – 57157 – 9 
\item 
Physics for Scientists and Engineers, Prentice Hall 4th Edition, by Giancoli 
		Chapters:  13-16, 32-34, 17-20  
		ISBN: 0 – 13 – 149508 – 9 

\end{itemize}




\section*{Relation with other subjects}

PHY250 closes PHY300. 
% Fifth Section %%%%%%%%%%%%%%%%%%%%%%%%%%%%%%%%%%%%%%%%%%%

\section*{Grading Policy}

The breakdown of the weighting of the Total Score will be as follows:

\begin{enumerate}
\item  	Assignments (30 $\%$)
\item   Mid term (30$\%$)
\item 	Final project (40$\%$)

\end{enumerate}

The minimum grade to pass the subject is 60 \% 



\section*{Assignments}

\begin{itemize}
	\item Problem sets are due approximately every two weeks. 
	\item  Work summited late is worth a maximum of 20 \% total credit.
	\item Homework must be summited in Moodle.
  \end{itemize}
  

  \section*{Midterm}


	The midterm is a closed-book exam and consists of solving a set of exercises that cover the theory taught during the first six weeks of class.


  
  \section*{Course Final Project}
  
  The course final project will utilize numerical methods to approximate a physical
  system and resolve the motion and interaction of the objects in the simulation.
  The project is intended as an opportunity to put into practice and further
  develop the techniques learned in the course. 
  


\section*{Relevance/Statement}

It is important to keep up with the material, to study regularly at home (at least 2 hours for every hour in class) and to do as many problems as you can 
(don’t limit yourself to the assigned or recommended problems, or merely the problems that are due).  

	You are welcome to work with other students, so long as the aim is furthering your understanding of the concepts and problem solving techniques. I am 
	happy to help work through problems, either in office hours or in class. Just remember, doing a problem yourself is very different from watching another 
	person do so. If you work together on problem sets, be sure to provide your own solution to every problem proving that you understand your writing. 
	In addition, some exam questions may be a resemblance to homework questions, so you’re encouraged to fully understand what you turn in. 
	Again, reading the relevant book sections before the material is taught is highly recommended. 


\subsection*{Last Day to Withdraw:   Tuesday 11/2/2021}

\vspace{5 mm}

\section*{Academic Integrity Policy}

Academic dishonesty in any form will not be tolerated in this course. Cheating, copying from any sources (including current or past students work, online sources 
or books), plagiarizing, or any other form of academic dishonesty (including doing someone else’s individual assignments) will result in, at the extreme minimum,
 a zero on the assignment in question, and could result in a failing grade in the course or even expulsion from DigiPen. Assisting others in cheating is prohibited 
 and will be equally punished.

\section*{ Special Considerations Support Services}
Students that have special needs due to medical issues, can apply for formal accommodations. 
The accommodations are student specific and are focused on helping the student to complete the learning 
process and achieve the goals in the course. Students that apply for accommodations for the first time should 
contact the Administration Office at 94 6365163 in order to start the process. Students that have already contacted the 
Administration Office will be informed about the general considerations through their Academic Advisor. Additionally, 
students should talk to the teacher in order to be informed about the details of the accommodations in this particular course.

%\subsection*{Class Structure}

%Bla bla bla ...

%\bigskip

%Bla bla bla ...

%\subsection*{Assessments}

%...

%\subsubsection*{Lecture}
%Bla bla ...

%\subsubsection*{Lab}
%...

%\subsubsection*{Final Exam and Class Project}

%Bla bla \textbf{Bla bla}.

%\subsection*{Grading Policy}
%The typical NC State grading scale will be used. I reserve the right to curve the scale dependent on overall class scores at the end of the semester. Any curve will only ever make it easier to obtain a certain letter grade. The grade will count the assessments using the following proportions:
%\begin{itemize}
%	\item \underline{\textbf{30\%}} of your grade will be determined by 2 in class midterm exams (15\% each).
%	\item \underline{\textbf{5\%}} of your grade will be determined by ...
%	\item \underline{\textbf{5\%}} ...
%    \item \underline{\textbf{10\%}}  ...
%	\item \underline{\textbf{15\%}} ...
%	\item \underline{\textbf{15\%}} ...
%\end{itemize}

% Add a figure %%%%%%%%%%%%%%%%%%%%%%%%%%%%%%%%%%%%%%%%%%%

%\begin{figure*}
%\includegraphics[width=1.3\textwidth,angle=90]{Concept_map_315.pdf}
%\end{figure*}

% Fifth Section %%%%%%%%%%%%%%%%%%%%%%%%%%%%%%%%%%%%%%%%%%%

%\newpage



% Course Schedule %%%%%%%%%%%%%%%%%%%%%%%%%%%%%%%%%%%%%%%%%%%

\section*{Outline and Tentative Dates}

\newcounter{week}
\setcounter{week}{1}

\raggedright
\begin{center}

 \begin{tabular}{|c |l| l| l|} 


 \hline
Timeline & Topic & HW &\multicolumn{1}{p{3cm}|}{ Approximate book Section} \\ [0.5ex] 
 \hline\hline




Week   \theweek &\multicolumn{1}{p{10cm}|}{ 
Static Fluids: Density, Pressure and Pascal Principle,
Buoyancy and Archimedes Principle} 
 &  & Ch 13 \\ 

 \hline
 Week  \stepcounter{week} \theweek &\multicolumn{1}{p{10cm}|}{ Fluids in Motion, Equation of  Continuity, Bernoulli's equation}&  Assig. 1 &Ch 13  \\ 
\hline
 Week  \stepcounter{week} \theweek &\multicolumn{1}{p{10cm}|}{Surface Tension, Viscosity, Poiseuille's equation, Drag forces. Generalization: Navier-Stokes equations}&   &Ch 13  \\ 

\hline
 Week  \stepcounter{week} \theweek & \multicolumn{1}{p{10cm}|}{Periodic Motion: Simple Harmonic Motion}
 & & Ch 14 \\ 
 \hline
 Week  \stepcounter{week} \theweek &\multicolumn{1}{p{10cm}|}{ Energy of SHM, Potential Diagram. Simple Pendulum.  Physical Pendulum.
Oscillations in 2D, DHM, Resonance} & Assig. 2 &Ch 14  \\ 




 \hline
 Week  \stepcounter{week} \theweek &\multicolumn{1}{p{10cm}|}{ Waves: Wave Motion,  Mathematical Representation 
Traveling Waves, Longitudinal and Transversal Waves}&  & Ch 15 \\ 



 \hline
 Week  \stepcounter{week} \theweek &\multicolumn{1}{p{10cm}|}{ Energy on a Wave. Principle of Superposition.
Interference, Standing Waves}  & Assig. 3 & Ch 15 \\ 

\hline
Week  \stepcounter{week} \theweek &\multicolumn{1}{p{10cm}|}{\textbf{Review and Midterm}}&  &  \\ 



 \hline
 Week  \stepcounter{week} \theweek & \multicolumn{1}{p{10cm}|}{Sources of sound: Resonance. 
Sound I: Intensity (dB), Sound Level, Ear Response. Beats. Sound II: Doppler Effect, Shocks Waves.} & Assig. 4 & Ch 16 \\ 
 \hline


 Week  \stepcounter{week} \theweek &\multicolumn{1}{p{10cm}|}{The Nature and propagation of light. } &   & Ch 16, Ch 32 \\ 
	 \hline

 Week  \stepcounter{week} \theweek &\multicolumn{1}{p{10cm}|}{
Geometric Optic: Light Reflection, Light Refraction, Snell's Laws, Diffraction. } &  Assig. 5  & Ch 16, Ch 32 \\ 
 \hline

 Week  \stepcounter{week} \theweek &\multicolumn{1}{p{10cm}|}{
  Physics of shading: Interaction of light with different materials } &    &  \\ 
   \hline

   Week  \stepcounter{week} \theweek &\multicolumn{1}{p{10cm}|}{
    Physically Based Rendering: The microfacet model, The reflectance equation } &  Assig. 6  &  \\ 
     \hline

 Week  \stepcounter{week} \theweek & \textbf{Review and Final Project}&  &  \\ 
 \hline
\end{tabular}

\end{center}


[This entire syllabus, particularly the time line, may be adjusted or changed at any time by the instructor.]

\end{document}


