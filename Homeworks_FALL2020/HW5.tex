% Don't touch this %%%%%%%%%%%%%%%%%%%%%%%%%%%%%%%%%%%%%%%%%%%
\documentclass[12pt]{article}
\usepackage{fullpage}
\usepackage[left=1in,top=1in,right=1in,bottom=1in,headheight=3ex,headsep=3ex]{geometry}
\usepackage{graphicx}
\usepackage{float}
\usepackage{array}


\newcommand{\blankline}{\quad\pagebreak[2]}
%%%%%%%%%%%%%%%%%%%%%%%%%%%%%%%%%%%%%%%%%%%%%%%%%%%%%%%%%%%%%%

% Modify Course title, instructor name, semester here %%%%%%%%

\title{PHY 250: Homework 5}
\author{Fall 2020}
\date{}

%%%%%%%%%%%%%%%%%%%%%%%%%%%%%%%%%%%%%%%%%%%%%%%%%%%%%%%%%%%%%%

% Don't touch this %%%%%%%%%%%%%%%%%%%%%%%%%%%%%%%%%%%%%%%%%%%
\usepackage[sc]{mathpazo}
%\linespread{1.05} % Palatino needs more leading (space between lines)
\usepackage[T1]{fontenc}
\usepackage[mmddyyyy]{datetime}% http://ctan.org/pkg/datetime
\usepackage{advdate}% http://ctan.org/pkg/advdate
\newdateformat{syldate}{\twodigit{\THEMONTH}/\twodigit{\THEDAY}}
\newsavebox{\MONDAY}\savebox{\MONDAY}{Mon}% Mon
\newcommand{\week}[1]{%
%  \cleardate{mydate}% Clear date
% \newdate{mydate}{\the\day}{\the\month}{\the\year}% Store date
  \paragraph*{\kern-2ex\quad #1, \syldate{\today} - \AdvanceDate[4]\syldate{\today}:}% Set heading  \quad #1
%  \setbox1=\hbox{\shortdayofweekname{\getdateday{mydate}}{\getdatemonth{mydate}}{\getdateyear{mydate}}}%
  \ifdim\wd1=\wd\MONDAY
    \AdvanceDate[7]
  \else
    \AdvanceDate[7]
  \fi%
}
%\usepackage{setspace}
\usepackage{multicol}
%\usepackage{indentfirst}
\usepackage{fancyhdr,lastpage}
\usepackage{url}
\pagestyle{fancy}
\usepackage{hyperref}
\usepackage{lastpage}
\usepackage{amsmath}
\usepackage{layout}

\lhead{}
\chead{}
%%%%%%%%%%%%%%%%%%%%%%%%%%%%%%%%%%%%%%%%%%%%%%%%%%%%%%%%%%%%%%

% Modify header here %%%%%%%%%%%%%%%%%%%%%%%%%%%%%%%%%%%%%%%%%
%\rhead{\footnotesize Text in header}

%%%%%%%%%%%%%%%%%%%%%%%%%%%%%%%%%%%%%%%%%%%%%%%%%%%%%%%%%%%%%%
% Don't touch this %%%%%%%%%%%%%%%%%%%%%%%%%%%%%%%%%%%%%%%%%%%
\lfoot{}
\cfoot{\small \thepage/\pageref*{LastPage}}
\rfoot{}

\usepackage{array, xcolor}
\usepackage{color,hyperref}
\definecolor{clemsonorange}{HTML}{EA6A20}
\hypersetup{colorlinks,breaklinks,linkcolor=clemsonorange,urlcolor=clemsonorange,anchorcolor=clemsonorange,citecolor=black}

\begin{document}

\maketitle

%\blankline

%\begin{tabular*}{.93\textwidth}{@{\extracolsep{\fill}}lr}

%%%%%%%%%%%%%%%%%%%%%%%%%%%%%%%%%%%%%%%%%%%%%%%%%%%%%%%%%%%%%%

% Modify information %%%%%%%%%%%%%%%%%%%%%%%%%%%%%%%%%%%%%%%%%
%E-mail: \texttt{anabela.turlione@digipen.edu}  \\

 %Office Hours: M 10-11:45am  &  Class Hours: T/Th 3-4:15pm \\

 %Office: ... & Class Room: ... \\
%% & \\
%Lab Room: ... & Lab Hours: W 3-5pm \\
%&  \\
%\hline
%\end{tabular*}

%\begin{figure*}
%\includegraphics[width=1.3\textwidth,angle=90]{Concept_map_315.pdf}
%\end{figure*}

\begin{center}
Death-line: November  24th    
\end{center}
\hrule



% First Section %%%%%%%%%%%%%%%%%%%%%%%%%%%%%%%%%%%%%%%%%%%%






\section*{Exercise 1}


A guitar string is supposed to vibrate at $247 Hz$, but is measured
to actually vibrate at $255 Hz$. By what percentage should the
tension in the string be changed to get the frequency to the
correct value?


\section*{Exercise 2}

Two strings on a musical instrument are tuned to play at
$392~Hz$ (G) and $494~Hz$ (B). (a) What are the frequencies of
the first two overtones for each string? (b) If the two strings
have the same length and are under the same tension, what
must be the ratio of their masses$ m_G/m_Al$ (c) If the
strings, instead, have the same mass per unit length and are
under the same tension, what is the ratio of their lengths
$\ell_G/\ell_A$ (d) If their masses and lengths are the same, what
must be the ratio of the tensions in the two strings?

\section*{Exercise 3}

Estimate the average power of a water wave when it hits the
chest of an adult standing in the water at the seashore.
Assume that the amplitude of the wave is $0.50 m$, the wavelength
is $2.5 m$, and the period is $4.0 s$.


\section*{Exercise 4}
The displacement of a bell-shaped wave pulse is described
by a relation that involves the exponential function:

\begin{equation}
D(x,t)=Ae^{-\alpha(x-vt)^2}
\end{equation}

where the constants $A=10.0$ m, $\alpha=2.0$ m$^{-2}$, and $v=3.0$ m/s. (a) Over the rage $-10.0 m \leq x \leq10.0 m$,
 plot D{x, t) in Octave at each of the three times t = 0, t = 1.0, and t = 2.0 s. Do
these three plots demonstrate the wave-pulse shape shifting
along the $x$ axis by the expected amount over the span of
each $1.0$~s interval? (b) Repeat part (a) but assume
$D(x, t) = Ae^{-\alpha(x+vt)^2}$.



%%%%%%%%%%%%%%%%%%%%%%%%%%%%%%%%%%%%%%%%%%%%%%%%%%%%%%%%%%%%%%

\end{document}


%%%%%%%%%%%%%%%%%%%%%%%%%%%%%%%%%%%%%%%%%%%%%%%%%%%%%%%%%%%%%%